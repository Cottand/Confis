\chapter{Conclusions}\label{ch:conclusions}

This project has introduced:
\begin{itemize}
    \item Confis, an accessible specification language for legal agreements, implemented as an internal DSL~\cite{fowlerDsl} using Kotlin~\cite{kotlinLang} as a host language.
    \item An intelligent and language-aware editor equipped with all the utilities of the IDE of a general-purpose programming language, implemented as an IntelliJ~\cite{intelliJRepo} editor.
    \item Conversion from Confis back to legal prose in plain English, encoded in Markdown and rendered as a live preview side-to-side with the editor.
    \item A rules engine that generates normative rules from a Confis agreement and a complex query, accompanied by a graphical interface that allows assembling such queries and rendering query results in plain English
    \item A user guide as an online website to complement the above tools.
\end{itemize}

% TODO

Confis is novel in that it sets a precedent for reconciling rule-based formalisms for legal agreements with accessible industry-oriented software.
We find that it largely succeeds at introducing a machine-readable formalism for representing contracts (aided by novel features such as graphical query UIs and converting a formalism back to plain English), but at the cost of compromising on the expressiveness of its language and the contracts it is able to represent.
This lack of expressiveness stems primarily from its lack of temporal logic and its lack of ability to deal with contract violations after they happen.


\section{Future Work}\label{sec:future-work}

Future work could have two main directions: improving the existing Confis language semantics and software prototype, and building upon the querying engine in order to programmatically deal with legal agreements.

\subsection{Better Language Semantics}\label{subsec:future:better-language-semantics}

This project chose to compromise on that area for the sake of a leaner learning curve -- but it does not conclude new abstractions with stronger guarantees are not possible.

\subsubsection{More Circumstances and Temporal logic}\label{subsubsec:future:more-circumstances-and-temporal-logic}
A better version of Confis that does not compromise on Accessibility may be possible with more Circumstance implementations, as well as perhaps integrating time and causality in the IR\@.

New Circumstances could include geographical location, \emph{force majeure}~\cite{forceMajeureDefinition}, repeated time periods, or possession of assets.

\subsubsection{An `On Breached' Allowance}

As discussed in~\autoref{subsubsec:limits-violations}, Confis struggles to deal with planning for a scenario after the contract has been breached.
This stems from the fact that a scenario can only stem from an \texttt{Allow}~\nameref{def:allowance}, but breaches fall under the~\texttt{Forbidden} state space.

Addressing this limitation could be possible by introducing a fifth type of Allowance that would account for the intersection of these two possible states.

\subsection{A Query API}\label{subsec:future:query-api}
One of the key contributions of the project is the rule-based querying that allows asking a contract complex questions like \emph{`Under what circumstances may I access this data set?'}.
While Confis makes a strong focus on human interaction through accessible graphical interfaces, it has a binary interface that makes the prototype usable in any JVM-based application~\cite{venners1998java}.


We envision distributed systems where services query and act according to the legal capabilities of their organisations given the circumstances they find themselves in.
An example would be dataset access control and GDPR for an organisation that holds sensitive personal data: services would be able to update and process data while aware of what the organisation is and is not allowed to do it.
Similarly, employees' access to customers' data can be automatically permissioned depending on how or why they are accessing teh data, or which customer's data they are tampering with.

The notion of services acting following the specifications of a legal agreement is in line with Szabo's original vision of smart contracts~\cite{szabo1997smart-contracts}, much like Knottenbelt's work on contract-driven agents~\cite{knottenbeltContractDriven} -- except Confis aims for non-logicians to write and review the contract specifications.

\subsection{An Implementation for Public-Key Signing Infrastructure}\label{subsec:future:signing}

Digital signatures~(see~\autoref{subsec:crypto:pubkey}) are key to verifying the identity of parties of a contract.
While Confis has all of the traits of a Ricardian Contract (see~\autoref{def:ricardian}) the public-key signing infrastructure is not implement by this project's prototype.

Developing such infrastructure could be a first step towards persisting verifiable Confis agreement metadata in a blockchain -- which in turn also implements Proof Of Existence, much like what Express Agreement provides~\cite{expressAgreement}.
