% Chapter Template

\chapter{Ethical Considerations}\label{ch:ethical}


\paragraph{Possible Discrepancies Between Machine-Readable Representations and Legal Requirements}
\label{par:legal-discrepancies}

This project provides new technology to understand and reason about a legal contract.
When it is applied in industry, there is always a risk that the contract the drafter means to represent (traditionally written as plain natural language) is not aligned with the contract represented by the Confis specification they actually write.
This can stem primarily from misunderstandings of the limitations of what Confis can represent (these are discussed in~\autoref{subsec:confis-lang-limits}).

\paragraph{Liability}\label{par:liability}

Additionally, it is possible that instead of being used to draft new contracts, this project is used to translate existing contracts into a formalism.
This creates the additional risk that a party may be subject to a natural language contract while checking their legal capabilities and obligations within the scope of the Confis agreement.
If these two different representations have discrepancies, then such a Party might inadvertently breach a clause in the original natural language representation.

It is the opinion of this project that the liability of this sort of legal blunder should lie with the party that agreed to be subject to a natural language contract but did not follow legal advice concerning that representation of the contract.
This problematic scenario is one of the reasons Confis makes an emphasis on authoring contracts, rather than translating existing agreements into a formalism.

\paragraph{Legal Validity}\label{par:legal-validity}

If this project wants to represent or encode real legal agreements, the representations should qualify as such within the legal system they intend to be used within.

Ideally, a court should be able to recognise a Confis representation as well as a
traditional natural language one.
Parties using Confis agreements should double-check their natural language equivalents qualify as contracts within their legal jurisdiction.
The background research this project has undertaken has verified this is commonly the case for Common Law jurisdictions~(which includes the United Kingdom)~\cite{larsonContractLawIntro, contractDef2018precedent, contractDefinition}.

\paragraph{Copyright}\label{par:copyright}

The prototype for this project makes use of several licensed, open-source software
libraries.
The licenses used are:
\begin{itemize}
    \item \textbf{Apache License} Used by the Kotlin Language~\cite{kotlinLang}, IntelliJ IDEA~\cite{intelliJRepo} and Gradle~\cite{gradleDSL}.
    \item \textbf{MIT License} Used by Easy Rules~\cite{easyRules}.
    \item \textbf{MSD 2-Clause Simplified License} used by MkDocs~\cite{mkDocs}
\end{itemize}

All of these licenses allow the distribution of modified and larger works under different licenses.
Confis includes copyright and license notices for all the above licenses, therefore it is compliant with all the licenses it has been granted.

\paragraph{Misuse}\label{par:misuse}

This project does not envision malicious use of the technology it introduces.
It is also not particularly vulnerable to remote attack, since data is not transmitted through any networks.


\paragraph{Data Privacy and Compliance}\label{par:data-privacy-compliance}

Legal agreements between businesses can be confidential both in their contents and their existence -- this project preserves these properties in that it does not upload or process any Confis document.
Contracts are drafted with an offline editor, and the software prototype process files offline.
Rules evaluation results are also not uploaded to the internet.

Therefore, Confis is also General Data Protection Regulation (GDPR)~\cite{gdprInfo} compliant.

\paragraph{Dual Use}\label{par:dual-use}

This project does have military applications: automating workflows around legal agreements could improve supply chains, including military ones;
but its focus and motivations are exclusively civilian.

Therefore, this project does fall under the definition of Dual Use.

\paragraph{Environmental Implications}\label{par:environmental-implications}

This project does not have major environmental implications.
