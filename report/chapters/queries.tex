\chapter{Queryable Documents}\label{ch:queries}

\section{Developing a suitable representation}\label{sec:queries-representation}

\subsection{Requirements}\label{subsec:queries-requirements}

% TODO find a source on why we would want this specifically out of Confis
Assuming an existing document, being able to perform the following operations on said document is desirable: we will call these operations \emph{queries} (or questions) made to the contract.
In order to be accessible, these must be intuitive and should not require deep

% TODO justify this being common
\paragraph{Querying for Legal Capabilities} A common use-case is for a party to want to figure out their legal capabilities with respect to a contract, as well as the capabilities of other parties.
Take the example of a tenancy agreement: a tenant may want to know whether they are allowed to have pets, or whether the landlord is allowed to enter their property.

Therefore, a successful query system should be able to provide answers to questions such as \textit{`May $A$ do $X$?'}

A party may also have some requirements to be able to perform some action - such as performing a payment.
Continuing the example of the tenancy agreement, a landlord may be allowed to enter the premises in case of emergency, but not otherwise.
Thus, a more general question could be `Under what condition may $A$ do $X$?'

\paragraph{Compliance verification} If figuring out a party's legal capabilities is a key part of dealing with a contract, so is figuring out their legal obligations.
Unlike a legal capability question, a compliance question cannot be formulated as \textit{`May $A$ do $X$?'} -- they should instead be along the lines of \textit{`What does X need to do in order to be compliant?'}.
We should also take into account that a party may have already done something to be compliant at the time of performing the query -- therefore we need to include some `state-of-the-world' in our question.
    Additionally, parties usually interact and actions between more than a single party may be needed to achieve compliance.

Therefore, we can generalise a compliance verification question to \textit{`Given a series of past events S, what actions need to take place in order for the contract to be complied with?'}.


\subsection{Confis Internal Representation}\label{subsec:confis-ir}
As~\cite{knottenbeltContractDriven} notes, there is a clear compromise to be made between how complex a contract representation is, and how simple the computations needed to process it are.

Contrary to solutions discussed in~\autoref{sec:nlp} such as~\cite{sleimi2018NLP4}, Confis does not try to generalise a contract into a set of normative rules (as defined in~\autoref{eq:basic-rule}) straight away.
Instead, it tries to preserve all the information that goes into assembling a contract into an Intermediate Representation (that we will call \textbf{\emph{Confis IR}}).
The Confis IR is then converted into different sets of rules depending on the query being performed.

The Confis IR is a set of data structures, mostly tuples and collections, that can be mapped to a JSON or protobuf schema for serialisation.
For the sake of brevity, this report does not contain the entire schema, but see for example~\autoref{fig:agreement-ir-tuple} for the top-level struct that makes up an agreement.

\begin{figure}[h]
    \centering
    \begin{minipage}{0.5\textwidth}
        \begin{minted}[autogobble]{kotlin}
            data class Agreement(
                val clauses: List<Clause>,
                val parties: List<Party>,
                val title: String,
                val introduction: String,
            )
        \end{minted}
    \end{minipage}
    \caption{Code for the top-level Agreement struct of the IR}
    \label{fig:agreement-ir-tuple}
\end{figure}



\section{Rule Generation}\label{sec:rule-generation}

Given Confis evaluates different rules depending on the query from the IR, rather than generating rules from the DSL, generating rules from the IR is a critical part of Confis.



% TODO examples of DSL -> IR and IR -> rules


\section{Tooling for Accessible Queryable Contracts}\label{sec:tooling-for-accessible-queryable-contracts}

\section{Confis as a Generalisation of Ricardian Contracts}\label{sec:generalisation-of-ricardian-contracts}
