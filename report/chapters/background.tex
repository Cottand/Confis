\chapter{Background}\label{ch:background}


\section{Legal Agreements}\label{sec:licensing}

A legal agreement is a fairly broad definition which varies with each country's legal system.
Most generally, it involves~\cite{contractDef2018precedent}:
\begin{itemize}
    \item An offer by a party
    \item An acceptance by at least one other party
    \item Intention to create a legal relationship
    \item Consideration for the offer -- a sort of `promise' where a price is paid for the offer (monetary or
    otherwise).
\end{itemize}

A licence is a specific kind of legal agreement which allows a party to perform an activity under some specific
conditions\cite{licenseDefinition}.
In order to limit the scope, this project will only consider licences (rather than legal agreements as a whole).

\subsection{Dataset Licensing}\label{subsec:licensing:dataset}
See~\cite{seismicDataLicence} for an example of a licence allowing access to data by a library.
The licence includes the following clauses, cherry-picked and simplified here for clarity:
\begin{itemize}
    \item The licence is non transferable.
    \item The data cannot be sold or used to provide a service
    \item The licensee can provide access to the data to third parties provided they sign a confidentiality agreement.
    The licensee becomes liable for any breach of the licence by such third party, so it is in the licencee's interests
    that the third party also complies to the licence.
    \item The data cannot be copied or modified for any other use other than the licensee's.
\end{itemize}

See a licence by The Economist Group~\cite[\textsection2.1]{economistIU2016licence} for an example
where a clause enforces confidentiality of the terms of the agreement, not just of the data
being shared.

\subsection{Software Licensing}\label{subsec:licensing:software}
See~\cite{jetbrainsEduLicence} for an example of a software licence between a software company (JetBrains) and users for
the use of their development software.
The licence includes the following clauses, cherry-picked and simplified here for clarity:
\begin{itemize}
    \item The licence is non-transferable.
    \item The software must be used only for specific but vague purposes \textcite[for non-commercial, educational purposes
    only]{jetbrainsEduLicence}.
    \item Only the licensee may use the software, but they may install it in multiple machines simultaneously.
    \item The software cannot be sold, rented, or modified.
    \item The licensee may provide feedback - if they choose to do they themselves grant JetBrains a very permissive
    licence to use it.
    \item There is a number of exceptions where JetBrains can revoke access - but in general, they may at any point
    terminate the agreement without a specific reason.
    \item The user may choose to use an early access development version of the software, subject ot its own separate
    licence.
\end{itemize}

While this contract is not machine-readable, JetBrains does have mechanisms in place to enforce its interests.
Their software comes with a launcher, \textit{JetBrains Toolbox}~\cite{jetbrainsToolbox} that controls access to it
depending on the licence status of the user.
This is most likely implemented in a centralised server controlled by JetBrains where it keeps records
of their licensees.

For a licensee, this means they cannot easily prove they did enter an agreement with JetBrains if JetBrains chooses
to deny issuing a licence~\citeTODO.


\section{Previous Work on Machine-Readable Contracts}\label{sec:machine-readable-contracts}

In the context of this project, \emph{machine-readable}~\cite{cambridgeMachineReadable, openDataMachineReadable} refers to an encoding for a formalisation of a legal agreement which can be processed by a suitable program.
The processing must relate specifically to uses of legal documents (therefore formats like Microsoft Word, plain UTF encoded text, or PDF do not qualify).

See also~\autoref{sec:nlp} for normative rule generation from legal text, and~\autoref{subsec:reasoning-with-legal-agreements} for reasoning with existing formalisms.

\subsection{Smart Contracts and Ethereum}\label{sec:ethereum}

Smart contracts~\cite{szabo1997smart-contracts}, where a protocol formalises a secure agreement or program over a
network, are supported to some extent in the first blockchain application, Bitcoin~\cite{nakamoto2008bitcoin}~(see~\ref{sec:bitcoin}) through scripting~\cite{bitcoinwiki-scripts}.

A bitcoin script is a list of instructions in the \textit{Script} language recorded with each transaction
(see~\nameref{subsec:btc:txs}) that gets executed by every participant that verifies that block and describes how the
payee can access the coins being transferred.
Bitcoin's Script has several limitations, such as lack of Turing-completeness (not all computable functions can be
represented in Script), value blindness (a script cannot easily specify an exact amount to be
withdrawn) and lack of internal state (beyond whether a transaction output is spent or
unspent)~\cite{buterin2015ethereum}.

Ethereum~\cite{buterin2015ethereum} is also a blockchain sharing many of Bitcoin's features, such as transactions and
a proof-of-work consensus algorithm.
While it also as scripting support, it varies greatly from Bitcoin's in that its built-in programming language is
Turing-complete, value-aware and allows keeping internal state.
For more details on Ethereum's scripting language, see~\ref{subsubsec:evm}.
The name of the coin of the Ethereum ecosystem is \textit{ether}.

Ethereum's state is encoded in \textit{accounts} - objects with their own address that contain the account's
balance~(in ether), the contract code~(if present), the nonce~(a transaction counter), and the account's
storage~(a persistent key-value store which is initially empty).

Ethereum introduces an application of the concept first presented by Szabo in~\cite{szabo1997smart-contracts}, in that it allows defining self-enforceable relationships between entities.
It is implemented thanks to the Ethereum Virtual Machine.

\subsubsection{The Ethereum Virtual Machine}\label{subsubsec:evm}

Ethereum's built-in scripting language is a stack-based low-level bytecode language called
\textit{EVM language}, and runs on an execution environment called the \textit{Ethereum Virtual
Machine} (EVM).
Besides its stack, the EVM does have memory (an infinitely expandable byte array) which resets after
each computation.
For persisting data, the account's storage is used~\cite{buterin2015ethereum}.

There are several high level languages that compile to EVM bytecode that are used to write smart
contracts, such as Solidity~\cite{solidity_repo} and Vyper~\cite{vyper_repo}~(the most active and
maintained languages at the time of writing~\cite{eth_smartContractLangs}).

% TODO code an examples if you're going to be writing on these

An application built on top of decentralised smart contracts is commonly called a
\textit{Decentralised Application} (or DApp)~\cite{masteringEthGlossary}.

\subsection{Juro}\label{subsec:juro}
Juro~\cite{juroWhitepaper} is a start-up business seeking to make structured machine-readable data out of contracts.
It focuses on:
\begin{itemize}
    \item Enabling easy collaboration between contract authors.
    \item Auditing and keeping track of the \textit{workflow} of the agreement (version control of the documents'
    drafts, signatures, amendments, views, etc.)
    \item Improving the user experience of employees involved in legal agreements by integrating into the company's CRM
    \textit{(Customer Relationship Management)} software (such as new hires and the Human Resources department).
    \item Ease of use by making legal documents searchable.
\end{itemize}

Juro aims to achieve this by encoding the contract as a JSON document~\cite{jsonSpec}, written by end-users
with the assistance of a web graphical user interface (GUI).
This document keeps all version control information and can be rendered into a human-readable format or exported
to PDF~\cite{pdf2020spec} or Microsoft Word~\cite{microsoftWordWeb} formats.
The JSON document can be further processed to provide extra analytics and tools, like summarization statistics or
agreement renewal reminders.

Juro's whitepaper lists `self-executing' clauses as future work~\cite[p.~6]{juroWhitepaper}.
Therefore Juro does not attempt to make any contributions with respect to verification of compliance or enforcement of
contract clauses.

\subsection{The Accord Project}\label{subsec:accord}

The Accord Project~\cite{accordHomepage} is an open source Linux Foundation project with the aim of enabling
machine-readable and machine-executable legal agreements.

Contracts are written pre-signature in a human-readable markup language similar to Markdown~\cite{markdownSpec}
where typed variables can be embedded.
During drafting, functions can be defined with a Domain Specific Language (DSL) that allow to perform
computations~(such as compounding interest over a period of time).

After signature, more behaviour can be written in code to check information against a contract (like checking for
breach of contract, or what action to take in a specific situation).
See Figure~\ref{fig:accord-post-signature-code} for an example where the clause checks for the state of a
delivery~(whether it is timely, has passed inspection, etc).

\begin{listing}[th]
    \centering
    \small
    \begin{minted}[autogobble, fontsize=\small]{javascript}
    contract SupplyAgreement over SupplyAgreementModel {
      clause acceptanceofdelivery(request : InspectDeliverable) : Response {

        let received = request.deliverableReceivedAt;
        enforce isBefore(received,now()) else
          throw ErgoErrorResponse{
            message : "Transaction time is before the deliverable date."
          }
        ;

        let status =
          if isAfter(now(),
            addDuration(
                received,
                Duration{ amount: contract.businessDays, unit: days}
            )
          )
          then OUTSIDE_INSPECTION_PERIOD
          else if request.inspectionPassed
          then PASSED_TESTING
          else FAILED_TESTING
        ;
        return Response{
          status : status,
          shipper : contract.shipper,
          receiver : contract.receiver
        }
      }
    }
    \end{minted}
    \caption[Accord contract clause as code]{A contract clause encoded in Accord's \textit{Ergo} language,
        from~\cite{accordAfterSignatureCode}.}
    \label{fig:accord-post-signature-code}
\end{listing}

\textit{Ergo} code (the language in which Accord allows specifying contract behaviour~\cite{accordErgo}) can then be
deployed to a permissioned open-source enterprise-oriented blockchain called \textit{HyperLedger
Fabric}~\cite{hyperledgerFabric_repo, accordFabricDeployDocs} or to a NodeJS server~\cite{accordNodeJSDeployDocs}.

While the Accord project does also aim to represent machine-readable contracts, its representation
cannot be easily written by a non-software developer (unlike Juro's,~\autoref{subsec:juro}).

\subsection{Proof of Existence}\label{subsec:proof-of-existence}

Among other contributions that ease the process (not unlike Juro,~\autoref{subsec:juro}) of negotiating
and drafting a contract,~\cite[Express Agreement]{expressAgreement} implemented Proof of Existence,
where parties put a hash of a signed copy of a contract in the Bitcoin blockchain~(see~\autoref{sec:bitcoin}).

This allows irrevocable proof of agreement verifiable by third parties.

\subsection{Ricardian Contracts}\label{subsec:ricardian-contracts}

First introduced by~\cite{grigg2004ricardian, ricardianWeb}, Ricardian Contracts aim to encode a legal agreement as a human-readable document (handwritten legal prose) accompanied by machine-readable metadata.
Additionally, it is cryptographically signed (see~\autoref{subsec:crypto:pubkey}) by parties and uniquely identifiable;
giving it some extra cryptographic guarantees (for example, signatures cannot be plausibly denied and terms cannot be tampered with after signing).


\begin{definition}[Ricardian Contract]
    \label{def:ricardian}
    A Ricardian contract can be defined as a document that is~\cite[\textsection3.1]{grigg2004ricardian}:
    \begin{itemize}
        \item a contract offered by an issuer to holders,
        \item for a valuable right held by holders
        \item such valuable right is managed by the issuer,
        \item human-written,
        \item readable by programs,
        \item digitally signed,
        \item carrying public keys (necessary to verify signatures),
        \item uniquely identifiable through a hash digest.
    \end{itemize}
\end{definition}

Using human-written legal prose has the advantages of leading to faster dispute resolution, better enforceability and more transparency when operating within the classical framework of agreements -- for all purposes, there is no difference between a traditional agreement and a Ricardian contract in this area.

\subsubsection{Drawbacks}

While Ricardian contracts can represent any type of agreement, they do not actually try to capture the \emph{meaning} of a specific type of agreement.

For example, a Ricardian contract may capture all the necessary information related to a Tenancy Agreement in a format that may allow a machine to enquire about the monthly rent -- but such a machine would need to know how tenancy agreements are structured.
In short, Ricardian contracts must follow a template and it is instances of this template that are easily read and queried by machines.

\subsection{Reasoning with Legal Agreements}\label{subsec:reasoning-with-legal-agreements}

The process of representing or acquiring knowledge from a legal agreement is typically considered as a process of writing conditional rules with the following basic conditional structure~\cite{sartorLegalReasoning, ferraroLegalNLPSSurvey}:

\begin{equation}
    \label{eq:basic-rule}
    r: \text{IF} \  a_{1}, \dots ,a_n \ \text{THEN} \ c
\end{equation}

Where $r$ is the unique identifier of the rule, $a_1, \dots, a_n $ are the \emph{antecedent} representing the conditions (which includes the context under which the rule is created) of applicability of the norm, and $c$ is the \emph{conclusion} representing the effect of the norm.

Note how~\nameref{subsec:juro} does not try to capture norms at all, and~\nameref{subsec:accord} chooses to use the abstraction of a program rather than that of a set of rules.
NLP, on the other hand (discussed in~\autoref{sec:nlp}) struggles to extract such rules from existing documents.

\subsubsection{Contract-Driven Agents}\label{subsubsec:contract-driven-agents}

An attempt at formalising a representation of the behaviours involved in a legal agreement and the interactions between the parties to such agreements contract-driven agents as presented in~\cite{knottenbeltContractDriven}.

The formalisation involves establishing a calculus based in multi-agents systems specifically oriented towards legal agreements, through the event calculus as first introduced by~\cite{kowalski1989logicEventBased}~(a logic-based system).
Note how the event calculus can be generalised by~\autoref{eq:basic-rule}.

Unlike Ricardian contracts~(see~\autoref{subsec:ricardian-contracts}), contract-driven agents attempt to describe the behaviours of all parties as well as how they react to other parties' behaviours.
Therefore, unlike Ricardian contracts, they also capture the semantics of the agreement between all the parties.

\subsubsection{Symboleo}\label{subsubsec:symboleo}

\emph{Symboleo}~\cite{symboleo2020} is a formal contract specification Language.
Much like~\nameref{subsubsec:contract-driven-agents} it aims to capture the semantics of the agreement.
It does this by providing several abstractions relating to legal concepts, such as \emph{powers} and \emph{obligations}, and creating axioms around them.

Time and events through a model based on temporal logic~\cite{temporalLogic1984} which specifies time instances for events, time intervals for situations, and makes use of pre-state and post-state situations for events.

% TODO Solidity ref
Symboleo also hopes to provide an implementation-agnostic specification by transpiling to blockchain languages, such as Solidity~\cite{solidity_repo}~(see~\autoref{subsubsec:evm}) -- much like~\nameref{subsec:accord}.
At the time of writing, Accord has such an implementation in place while Symboleo does not.

An example of a contract (a meat sales agreement that can be found in~\autoref{tab:meat}) translated into Symboleo is given under the Appendix in~\autoref{fig:symbolio:meatSales}.


\section{Natural Language Processing and its Limitations}\label{sec:nlp}

Human-written text, which is how legal agreements are typically persisted\footnote{Note other formats like verbal agreements may also constitute legal agreements~\cite{larsonContractLawIntro}} can be converted into machine-readable form by attempting to translate the natural language in which they are written through Natural Language Processing~(NLP), which uses techniques such as symbolic reasoning and deep learning~\cite{GoldbergNLPReview}.

Existing technologies trying to process legal text -- such as~\cite{angeli2015NLP1},~\cite{lapata2016NLP2},~\cite{lapata2018NLP3}, or~\cite{sleimi2018NLP4} -- try to produce normative rules.

At the time of writing, they fall short of fully capturing all relevant information in an existing legal document -- therefore they lack the~\nameref{def:completeness} property discussed in~\autoref{sec:eval:goals}.
They face the main following challenges~\cite{ferraroLegalNLPSSurvey}:

\paragraph{Cross-referencing}

Legal text is structured into sections and subsections which contain sentences which represent clauses applicable in a specific context.
Documents therefore often reference information across sections or across documents.
Some of these references may not even be explicit: often, contracts have a `Definitions' section solely for the role of disambiguating terms which the reader is expected to use in their understanding of the document.

NLP solutions must therefore be able to deal with referencing in order to extract context and resolve lexical ambiguities.

\paragraph{Ambiguity}
While legal documents aim to remove ambiguity and ideally produce a single interpretation, unintended ambiguities arise from the use of natural language.

The most likely ambiguity is \emph{referential ambiguity} where a word or phrase has multiple meanings inferred from current context, from a parent statement (which is unique to legal text), or from other document sections via cross-referencing.

Natural language can also represent different logical interpretations, which is referred to as \emph{local ambiguity}.
An example of this is using the term ``and'' to represent a disjunction instead of a conjunction.

\paragraph{Sentence Complexity}

Sentences used in legal prose tend to be much longer than sentences from other domains.
The average number of lexical units in a sentence written in English in Wikipedia is about 19~\cite{wiki19words}, while sentences from a legal document can have more than 50~\cite{ferraroLegalNLPSSurvey} as well as have complex syntactic structures (see for example~\autoref{tab:geophys-long-clause}).

Current NLP technologies struggle with such sentences: syntactic analysers and predicate-argument extraction tools sometimes do no correctly capture the scope of coordinate conjunctions, nor the antecedents of subordinate phrases.

\begin{table}[h]
    \centering
    \setlength{\fboxsep}{10pt}
    \fbox{
        \begin{minipage}{0.8\textwidth}
            On termination of this Licence the Licensee shall cease to have any rights or Licence in
            respect of the Data or any adaptation thereof and shall cease to use the same and shall
            erase the Data or any adaptation thereof from any storage apparatus and shall return to
            the Library without demand the Data and all existing copies thereof made by the
            Licensee and shall warrant in writing to the Library that all data, plots, displays, results,
            analyses, variations and modifications derived from the Data are destroyed.
        \end{minipage}
    }
    \caption{Clause \textsection3.2 of~\citetitle{seismicDataLicence}~\cite{seismicDataLicence}}
    \label{tab:geophys-long-clause}
\end{table}

\paragraph{Normative Effects and Modalities}

Legal documents typically encode norms, and words specific to legal concepts such as \emph{right, power, immunity, liability, privilege}, etc significantly alter the way a sentence is interpreted.
Properly capturing modalities or behaviours such as a \emph{requirement} or a \emph{permission} is a major difficulty the state of the art of Artificial Intelligence in Law struggles to deal with~\cite{ferraroLegalNLPSSurvey}.


\section{Domain-Specific Languages}\label{sec:dsls}

The use-case of representing complex, formalised abstractions in a machine-readable formats is a common one, and general-purpose programming languages (Java, Python, etc) are designed with exactly this goal in mind.

A class of languages made to target a specific problem, rather than being general purpose, also exists: these are \emph{Domain-Specific Languages} (or \emph{DSLs})

According to~\cite{fowlerDsl}: ``A DSL is a computer language that is targeted to a particular kind of problem, rather than a general purpose language that is aimed at any kind of software problem''.
In order to write legal agreements and lower the barrier of entry as much as possible (ideally, no software engineering training should be required) a well-documented and easy-to-use DSL is a good solution: expressive enough to be able to represent a large range of agreements, but restrictive enough that language features should not require extensive training.

Several modalities of DSLs exist that can be used.
The compromises they make are in cost (for developing a framework for them), ease of writing, readability, and expressiveness~\cite{fowlerDslGuide, fowlerLangWorkbench}.

\paragraph{Textual Internal DSLs}

Textual Internal DSLs make use of a host language, and use that language constructs in particular ways to give the internal DSL a particular feel.
They share their syntax with and are a subset of their host language.
This solution is the easiest to develop but forces the DSL author to conform to a host language and to extend their DSL within the bounds of what the host language can do.

\paragraph{Textual External DSLs}

Textual External DSLs have their own custom syntax and therefore they need a full new parsed to be processed.
This solution provides the most flexibility in what the language can express but is more expensive in terms of development efforts for the DSL author.

\paragraph{Graphical DSLs}

While Textual DSLs require writing and parsing text, graphical DSLs can be achieved with a tool providing a graphical user interface that then builds some internal representation that can be processed.
This solution is the most accessible for users of the DSL with non-technical backgrounds but is also expensive in terms of development cost, as both an internal representation and a GUI builder must be developed.

% TODO revisit figure placement in this section :c

\subsection{Candidates for Internal DSL Host Languages}\label{subsec:dsl-host-candidates}

This project is concerned with making a DSL with low development costs, and as such it will consider several platforms as host language candidates for an internal DSL\@.
\begin{definition}[DSL Host Traits]
    \label{def:host-lang-requirements}
    Languages desirable to be host for DSLs usually have some of the following features:
    \begin{itemize}
        \item \textbf{Operator overloading} for easily including symbols familiar to most users in the language, such as \texttt{`+'}, \texttt{`-'}, \texttt{`\&'}, etc.
        \item \textbf{Infix functions} for constructing sentence-like expressions or statements, such as \texttt{`if car is parked'}.
        \item \textbf{Scripting facilities} that allow compiling a small text snipped without having to define a full program that includes an entrypoint.
        \item \textbf{A Tooling Platform} which allows developing tools which enahnce the wiritng experience.
        Examples of this include GUI builders and IDE features (like syntax highlighting and inline compiler warnings).
        \item \textbf{Scripting}~(see~\autoref{subsec:scripting-support}) which allows parsing stand-alone text files in the host language that are interpreted by a DSL-aware host.
    \end{itemize}

\end{definition}

\subsubsection{Haskell}

Haskell~\cite{haskellDocs} is a functional programming language which can be compiled or interpreted.
It has advanced features including most of the ones mentioned in~\nameref{def:host-lang-requirements}.

These are demonstrated in~\autoref{fig:haskellBasicDsl}, where Haskell functions are used as commands and statements of the BASIC language.
This results in a valid Haskell expression (with some language modifications) that also defines a valid BASIC program -- therefore allowing to write and compile BASIC without actually needing a BASIC parser.

\begin{listing}[h]
    \centering
    \begin{minipage}{0.8\textwidth}
        \begin{minted}[
            autogobble,
            frame=lines,
            fontsize=\small,
            framesep=2mm]{haskell}
    {-# LANGUAGE ExtendedDefaultRules, OverloadedStrings #-}
    import BASIC

    main = runBASIC $ do
        10 GOSUB 1000
        20 PRINT "* Welcome to HiLo *"
        30 GOSUB 1000

        100 LET I := INT(100 * RND(0))
        200 PRINT "Guess my number:"
        210 INPUT X
        220 LET S := SGN(I-X)
        230 IF S <> 0 THEN 300

        240 FOR X := 1 TO 5
        250   PRINT X*X;" You won!"
        260 NEXT X
        270 STOP

        300 IF S <> 1 THEN 400
        310 PRINT "Your guess ";X;" is too low."
        320 GOTO 200

        400 PRINT "Your guess ";X;" is too high."
        410 GOTO 200

        1000 PRINT "*******************"
        1010 RETURN

        9999 END
        \end{minted}
    \end{minipage}
    \caption[BASIC program written in a Haskell DSL]
    {A BASIC program written in a Haskell DSL, from~\cite{haskellBasicDSL}}
    \label{fig:haskellBasicDsl}
\end{listing}

Haskell additionally permits defining the precedence of operators and infix functions, allowing to define the direction in which associativity propagates~\cite{haskellFixity}.
This can be useful to form English-like sentences in a DSL such as \texttt{`Bob did eat after Alice did eat'}, where \texttt{`after'} can be an infix function with lower precedence that \texttt{`did'}, which forms the other two sentences.

\paragraph{Tooling} Haskell tooling is comparatively lacking next to more mainstream general purpose programming languages (such as Kotlin) in that the communities developing its tooling platform are smaller.

\subsubsection{Groovy}

Groovy~\cite{groovyLangIndex} is a general purpose language with plenty of features specifically designed for DSLs.
One of the most used examples of such DSLs is the Gradle Build Language~\cite{gradleDSL} which specifies configuration for the Gradle build tool.
This sets the build tool apart from similar tools, like Maven and Makefiles, which do not use DSLs.
An example of a script for such a DSL is given in~\autoref{fig:gradleGroovyDSL}.

\begin{listing}[h]
    \centering
    \begin{minipage}{0.8\textwidth}
        \begin{minted}[
            autogobble,
            frame=lines,
            framesep=2mm]{groovy}
            plugins {
                id 'groovy'
                id 'application'
            }

            repositories {
                mavenCentral()
            }

            dependencies {
                implementation 'org.codehaus.groovy:groovy-all:3.0.9'
                implementation 'com.google.guava:guava:30.1.1-jre'

                testImplementation 'junit:junit:4.13.2'
            }

            application {
                mainClass = 'demo.App'
            }

            tasks.named('test') {
                useJUnitPlatform()
            }
        \end{minted}
    \end{minipage}
    \caption[Gradle Groovy Build script]
    {A Gradle Groovy build script, from~\cite{gradleDSL}}
    \label{fig:gradleGroovyDSL}
\end{listing}

Unlike to the other candidates in this section, Groovy allows for dynamic typing.
This means that the language offers more runtime flexibility when designing a DSL, as new variables and classes can be created as the script is interpreted.

\paragraph{Tooling} Groovy has feature-rich tooling that allows parsing standalone scripts with a DSL context, but its external tooling (like intelligent editors) is limited due to the nature of its runtime dynamic typing.
In the example of~\autoref{fig:gradleGroovyDSL}, an editor would not be able to suggest autocompletion for \texttt{`useJUnitPlatform()'} inside the configuration of the \texttt{test} task.

\subsubsection{Kotlin}\label{subsubsec:kotlinLang}

Kotlin~\cite{kotlinLang} is another general-purpose language.
The language supports infix functions and operator overloading.
An example DSL that uses Kotlin as the host language is given in~\autoref{fig:kotlinHtmlDsl}.

\begin{listing}[h]
    \centering
    \begin{minipage}{0.9\textwidth}
        \begin{minted}[
            autogobble,
            frame=lines,
            framesep=2mm]{kotlin}
        import com.example.html.*

        fun main(args: Array<String>) {
            val generated = html {
                head {
                    title {+"XML encoding with Kotlin"}
                }
                body {
                    h1 {+"XML encoding with Kotlin"}
                    p  {+"this can be used as an alternative to XML"}

                    // an element with attributes and text content
                    a(href = "https://kotlinlang.org") {+"Kotlin"}

                    // content generated by
                    p {
                        for (arg in args)
                            +arg
                    }
                }
            }
            println(generated.toString())
        }
        \end{minted}
    \end{minipage}
    \caption{Kotlin DSL for building HTML, from~\cite{kotlinTypeSafeBuilders}}
    \label{fig:kotlinHtmlDsl}
\end{listing}

\paragraph{Tooling} While the language is not as flexible as Haskell or Groovy, it is supported by well-maintained and feature-rich external tooling (in terms of the \textit{IntelliJ IDEA} editor) and has stand-alone scripting support~\autoref{subsec:scripting-support}.

% TODO add link on why we used this
Additionally, the editor IntelliJ IDEA supports live evaluation of Kotlin inside text boxes, allowing to provide hints at the same time it compiles a specific snippet of code~\cite{intelliJRepo}.

\subsection{Scripting Support}\label{subsec:scripting-support}

Depending on the DSL being developed, it may be necessary for the host language to provide scripting support.
By \emph{scripting support}, we mean allowing a program to evaluate a file or a string of text, at runtime, and share properties with the file.
We will call this file a \emph{script}.

In general, interpreted languages offer this functionality through \texttt{`eval(...)'} methods or similar.
They may not be necessarily well-suited for a DSL as they also tend to have weak or dynamic typing (this is the case of Python, for example).
Other compiled languages may offer this functionality:
\begin{itemize}
    \item Java offers it through JSR 223~\cite{javaScripting} and allows languages other than Java in the scripts.
    \item Kotlin~\cite{kotlinScriptKeep}.
    \item Groovy~\cite{groovyScripting}, which is widely used inside the Gradle build system (see~\autoref{fig:gradleGroovyDSL} for an example) where Gradle DSL scripts make up the configuration build files (in a way comparable to Makefiles).
\end{itemize}

For DSLs in particular, there is the requirement that the scripting host is aware of the DSL when evaluating the script~\cite{kotlinScriptKeep}.
This is what allows using a library-defined function in a stand-alone file without compiling errors.
Notice how this is not the case of the Haskell and Kotlin DSL examples in~\autoref{fig:haskellBasicDsl} and~\autoref{fig:kotlinHtmlDsl} (where imports and program entry-points are necessary).
% TODO section on the main differences between machine-readable representations of contracts?


\section{Language Tooling}\label{sec:language-tooling}

Most Integrated Development Environments (or IDEs) have support for \emph{extensions} or \emph{plugins} that allow extending their functionality~\cite{ideaExtensionPoints, vscodeExtensions}.
A common use-case for this is enhancing developer productivity by providing visual cues and aids to textual development by, for example, reporting compilation errors and warnings inside an editor.

Extensions also typically allow arbitrary graphical user interfaces inside the program window, with access to the contents of the files opened.

\subsection{Custom Scripting Support in IntelliJ IDEA}\label{subsec:scripting-in-intellij}

The IntelliJ IDEA editor~\cite{intelliJRepo} allows a specific extension point for providing script definitions~\cite{kotlinScriptKeep, ideaExtensionPoints} and enabling development features such as syntax highlighting, error reporting, and autocompletion inside custom-defined scripts.


\section{Rules Engines for the JVM}\label{sec:rules-engines}

This project requires both rule evaluation and making a Domain-Specific language with the same host language.
This rules out some logic programming languages appropriate for rules evaluation (namely, Prolog~\cite{prologBirth1988}) as well as Answer Set Programming solutions~(such as the Potassco Project~\cite{potassco}) because of their lack for DSL facilities~(see~\nameref{def:host-lang-requirements} for details on what some of the requirements are).

\subsection{Easy Rules}\label{subsec:j-easy-rules}

Easy Rules~\cite{easyRules} is a rules engine~\cite{fowlerRulesEngine} for Java (therefore it can also be used in~\nameref{subsubsec:kotlinLang}).

It allows programmatically generating rules (through \texttt{Rule} instances) as well as evaluating them.
This API is particularly swell-suited to this project because it allows defining the predicate that triggers rule execution (the \texttt{when} clause in~\autoref{fig:easyRulesProgranmmatic}) at runtime rather than at compile-time.

\begin{listing}[h]
    \begin{minted}[
        autogobble,
        frame=lines,
        framesep=2mm]{java}
                Rule weatherRule = new RuleBuilder()
                        .name("weather rule")
                        .description("if it rains then take an umbrella")
                        .when(facts -> facts.get("rain").equals(true))
                        .then(facts -> System.out.println("It rains, take an umbrella!"))
                        .build();
    \end{minted}
    \caption[Programmatic Example of a Rule from Easy Rules]{Programmatic Example of a Rule from Easy Rules, from~\cite{easyRules}}
    \label{fig:easyRulesProgranmmatic}
\end{listing}

Rules evaluation can be customised to add a notion of priority between rules (which might allow to create precedence between clauses in contracts).
Additionally, there is support for continous rule application, where rules are evaluated on known facts continously until no more rules are applicable.
This allows building a knowledge based through inference (rather than just evaluating rules sequentially).
