\chapter{A Domain Specific Language for Legal Agreements}\label{ch:lang}

One of the key focuses of this project is not just to be able to represent legal contracts with specific properties and capabilities as described in~\autoref{ch:queries};
but also to make it as easy as possible for people with non-technical backgrounds to use such representations.
Simply put, a lawyer should not need to learn JSON.\\

This is the main motivation for developing tooling which aims to make it easier for people of non-technical backgrounds to produce, modify, and understand Confis legal agreement representations.
The core of this tooling is the Confis language~(\autoref{sec:developing-a-dsl}) and an IDE-assisted editor~(\autoref{sec:additional-dsl-tooling}).


\section{Motivations For a DSL}\label{sec:developing-a-dsl}

\subsection{Implementation Requirements}\label{subsec:dsl:requirements}

The language should fulfil the core requirements set out in~\autoref{ch:evaluation}.
This involves prioritising~\nameref{def:accessibility} while making sure~\nameref{def:completeness} and~\nameref{def:soundness} are possible.

\subsubsection{Easy to write while still machine-readable}

Writing a Confis agreement should be close to writing plain English, while still being machine-readable.
For more details on the meaning of \emph{machine-readable} in the context of this project, see~\autoref{sec:machine-readable-contracts}.\\


A compromise must be struck between ease of writing and machine-readability.

\paragraph{Data Serialization Language}

On one extreme, a data serialization text file (like JSON, YAML, or XML) would allow writing text that can be easily parsed by a program.
But writing such files requires some data structures knowledge;
and because of their key-value nature they do not allow writing sentences, leaving them too far from the readability of human-written legal prose -- thus they lack the~\nameref{def:accessibility} property.

\paragraph{Natural Language Processing}

On the other extreme, legal prose processed through a language processing program allows the drafter to completely ignore the machine-readable aspect of the document.
Readers would be able to integrate such documents in their existing workflows -- as they would need to make no transition from their existing, non-machine-readable documents.
This is approach is discussed in~\autoref{sec:nlp} -- we conclude it does not meet the~\nameref{def:completeness} property.  \\

A compromise between these two solutions would be a language formal enough that it can be parsed by a program, but natural enough that natural language sentences can be recognised in it.
Python or AppleScript~\cite{Sanderson2010appleScript}~(see~\autoref{fig:appleScript}) are good examples of programming languages (and therefore parseable) that are engineered with the goal of resembling English as much as possible.

\begin{figure}[h]
    \centering
    \begin{minipage}{0.8\textwidth}
        \begin{minted}[
            autogobble,
            frame=lines,
            framesep=2mm
        ]{applescript}
                set the firstnumber to 1
                set the secondnumber to 2
                if the firstnumber is equal to the secondnumber then
                    set the sum to 5
                end if
        \end{minted}
    \end{minipage}
    \caption{Sample code snippet of the AppleScript Language~\cite{Sanderson2010appleScript}}
    \label{fig:appleScript}
\end{figure}

\subsubsection{Easy to develop and extend}

With pragmatic implementation efforts in mind, this project should not aim to develop its own parser and interpreter or compiler: the project hopes to be more concerned with introducing a suitable abstraction that allows both drafting and processing legal documents.

\subsubsection{Additional tooling for ease of use and correctness}

A key aspect of development for a new user of a language is to understand the concepts of syntax, compile errors, and invalid programs.
A great aid to developing this understanding are visual cues in editors in intelligent development environments (or IDEs).

For the Confis DSL to be successful, it must be easy for the drafter of legal agreements to reason about the correctness of the document within the formalisms set out by this project.
Good tooling is therefore a key requirement in order to achieve~\nameref{def:accessibility}.

\subsection{Implementation Requirements Conclusion}\label{subsec:dsl-design-conclusion}

The requirements of~\autoref{subsec:dsl:requirements} lead to the following design choices:

\paragraph{A Textual Domain-Specific Language} For a good compromise between readability, flexibility and rigidity of the agreements that can be written, this project therefore proposes developing a DSL~(see~\nameref{sec:dsls}) as a suitable compromise that allows working with a human-readable encoding which can then be compiled to a suitable machine-readable representation.

We will call this language \textbf{\emph{Confis DSL}} (or \emph{Confis language}) and the machine-readable representation it compiles to \textbf{\emph{Confis Internal Representation}} (or \emph{Confis IR}).
The Confis IR will be used for processing as discussed in~\autoref{ch:queries}.

\paragraph{An Internal DSL}
For the sake of development costs, we use a suitable host language to implement the Confis DSL (as opposed to developing an external DSL).

\paragraph{A separate Internal Representation}
Another key design decision is to develop the internal representation (the Confis IR) separately from the DSL\@.
By `develop separately' we mean that the internal representation should be enough to represent an agreement, and it should be possible to construct it independently of the Confis DSL\@.
% TODO discuss in future work
This should allow developing a stand-alone language as an external DSL (or alternatively, a graphical DSL or a new internal DSL in some other host language) in the future that utilises the same Confis IR -- thus preserving the formalisms this project contributes in this new language.

\paragraph{Kotlin as a host language} Several host languages can serve to build a DSL.
A few options are discussed in~\autoref{subsec:dsl-host-candidates}, like Groovy and Haskell.
The choice to use Kotlin stems from how feature-rich the surrounding tooling is -- such as its stand-alone custom scripting~\cite{kotlinScriptKeep} and IDE support~\cite{intelliJRepo}.\\

\section{Language Semantics and Design}\label{sec:language-semantics}

This section is concerned with the structure, semantics, and design decisions regarding the Confis DSL.
For the implementations details of the DSL, see~\nameref{subsec:dsl-implementation}.
For motivations and priorities of the project, see~\nameref{sec:eval:goals}.\\

Given decoupling the Confis language and the Confis IR is a requirement as discussed in~\autoref{subsec:dsl:requirements}, the la

% TODO come back after doing reasoning background
Confis introduces the following formalisms with the intention of restricting

\begin{definition}[Party]
    \label{def:party}
\end{definition}

\begin{definition}[Action]
    \label{def:action}
\end{definition}

\begin{definition}[Capability]
    \label{def:capability}
\end{definition}


\begin{definition}[Requirement]
    \label{def:requirement}
\end{definition}

\subsection{Language Implementation}\label{subsec:dsl-implementation}

This subsection is concerned with the implementation details of the Confis DSL.
For design decisions regarding the language (like what is a Sentence or the difference between a Requirement and a Capability) please see~\nameref{sec:language-semantics}.\\

Because of the nature of internal DSLs~(see~\autoref{sec:dsls}), the syntax of the Confis DSL must be a subset of the Kotlin language's.
% TODO add future work link for own lang
Given the Confis DSL and the Confis IR are decoupled this section will not go into too much detail on how the DSL is implemented as most of that is specific to the Kotlin language, and the DSL could have been implemented as an internal DSL with a different host language anyway (like Haskell or Groovy).
For more information on how DSLs can be built in Kotlin in general, see~\autoref{subsubsec:kotlinLang}, and in particular refer to~\cite{kotlinTypeSafeHTL}.

% TODO is this worth writing about at all

\section[The Confis Editor]{Additional Editing Tooling: the Confis Editor}\label{sec:additional-dsl-tooling}

One of the key requirements for Confis is~\nameref{def:accessibility}.
In order to achieve this the Confis framework includes an extension to the IntelliJ Development Environment~\cite{intelliJRepo} - the \textbf{\emph{Confis Plugin}}.
Bundled with this plugin is the \textbf{\emph{Confis Editor}}.\\

The Confis Editor has two main goals:

\paragraph{Providing context-aware aid for authoring agreements} Much like an IDE provides relevant aid to a software engineer (such as reporting compile errors before attempting to compile the program) the Confis Editor should allow an agreement drafter to write in the Confis DSL without needing to deal with command-line utilities or knowing the tooling around Kotlin language.

\paragraph{Providing a human-readable preview of the Confis Agreement} While a Confis Agreement is a strict formalism written in code, it is engineered to resemble English sentences.
The Confis Editor should display a document in legal prose, made up from text generated from the Confis agreement.
This preview should resemble a traditional legal document, and appear side-to-side the code the agreement drafter is writing in order to provide a useful live preview of the contract being authored.



