% Chapter Template

\chapter{Evaluation}\label{ch:evaluation}


\section{Confis Project Goals}\label{sec:eval:goals}

This project hopes to leverage software engineering techniques (like domain-specific languages), existing logic abstractions (like normative rules), and existing technologies relating to law (like Ricardian contracts) in order to provide a framework allowing people with non-technical and non-legal backgrounds to draft, review, and access information in legal agreements.

The following criteria are motivated by how existing technologies in the state-of-the-art (see sections~\ref{sec:nlp} and~\ref{sec:machine-readable-contracts}) fail to address them:

\begin{definition}[Meaningful Representation]
    \label{def:meaningful-representation}
    Confis should be able to fully represent a legal agreement in a legal context.

    That is, (albeit with the use of tooling) a Confis agreement should be usable in place of a legal one.

\end{definition}

\begin{definition}[Accessibility]
    \label{def:accessibility}
    Whatever the encoding of a Confis agreement, should be possible to produce (albeit with the use of tooling) such Confis agreement without an in-depth understanding of the formalism that encodes the agreement.
\end{definition}

The following are loosely defined -- this is unavoidable because this project aims to introduce a formalisation in legal documents, which are not well-defined.
Please see~\nameref{sec:language-semantics} for the definitions of~\nameref{def:action},~\nameref{def:party},~\nameref{def:capability}, and~\nameref{def:requirement}.

% TODO revisit after writing reasoning section
\begin{definition}[Completeness]
    \label{def:completeness}
    A Confis agreement $C$ is complete with respect to a legal agreement $L$ if it represents $L$ and:

    \begin{itemize}
        \item If $L$ allows a legal capability $c$, then a $C$ also allows $c$
        \item If $L$ forbids a legal capability $c$, then $C$ also forbids $c$
        \item If $L$ has a requirement $r$, then $C$ has $r$
    \end{itemize}
\end{definition}


\begin{definition}[Soundness]
    \label{def:soundness}
    A Confis agreement $C$ is sound with respect to a legal agreement $L$ if it represents $L$ and:

    \begin{itemize}
        \item If $C$ allows a legal capability $c$, then a $L$ also allows $c$
        \item If $C$ forbids a legal capability $c$, then $L$ also forbids $c$
        \item If $C$ has a requirement $r$, then $L$ has $r$
    \end{itemize}
\end{definition}

\section{Language Formalism Evaluation}

\section{Software Deliverables}\label{sec:software-deliverables}

\subsection{Language Implementation}

\subsection{Querying Engine}

\subsection{Tooling Implementation}

\section[Overall Evaluation]{Overall Evaluation With Respect to Goals}


\section{Machine Readable Licence Representations}\label{sec:licence-representations}

% we are looking to eavaluate the project - a lot of this should go into the intro probably...

% measures will be cost (smart contracts!), lawyer avoidance, and advice from people that know about this stuff

Other goals of this project require that a legal agreement can be encoded for it to become
structured data.
The project should come up with, or find in the literature and adapt, a suitable format for a legal
agreement.
In order to limit the scope of the project, we will narrow the agreement to be encoded to
licences.\\

A successful project should include:
\begin{itemize}
    \item \textbf{Machine-readable representation of licenses}.
    \item \textbf{Prototype Software} that uses the contract representation to ease the existing
    manual workflow related to agreements in a business.
    This could include a searchable database that is aware of licenses related to the same data or
    software~(see~\ref{sec:contract-registry}).
    The software should, to some extent, allow querying a contract to be able to determine the
    conditions which the license allows to use the data in.
    \item Secure, \textbf{self-executing clauses} that all parties bound to the contract can trust.
    This includes making it so parties cannot have plausible deniability or repudiation if they are
    dishonest.
\end{itemize}

Legal contracts are complex documents that may list vague or hard-to-encode conditions and
situations.
I expect not being able to (inexpensively) capture the entire meaning of a contract in a
machine-readable format.
This disassociation between reality and representation should be fully taken into account when
providing guarantees and making assumptions.
For example, a successful prototype should be aware of this enough to refer the user to the original
license when it is unable to provide certainties with respect to a query against a license.

In other words, this uncertainty created by attempting to represent a contract in a machine-readable
encoding should be part of the encoding itself, for the sake of completeness and usability.\\

Evaluation criteria would include whether the project can successfully encode a wide range of
licence agreements.


\section{A Universal Contract Registry}\label{sec:contract-registry}

A decentralised contract registry, shared by multiple businesses (not unlike a decentralised version
of Juro, see~\ref{subsec:juro}), can be a very useful contribution if it manages to meet the
requirement of privacy.
Businesses need not just the terms of their agreements to be secret to competitors
(see~\cite[\textsection2.1]{economistIU2016licence}), but also the existence of the contracts
themselves must remain private.
