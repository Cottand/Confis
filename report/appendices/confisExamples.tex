\chapter{Confis Code Samples}\label{ch:confis-examples}

\newenvironment{code}{\captionsetup{type=figure}}{}


\section{Confis Software Archive Code Extracts}\label{sec:confis-software-code-extracts}

The following is a stripped-down version of the \texttt{AgreementBuilder} class (the full version can be found in the software archive).
It is meant ot demonstrate how normal Kotlin functions can come together inside a single class that can be bound to a script in order to write agreements like that of~\autoref{fig:confis:minimal}.
\begin{code}
    \begin{minted}[
        autogobble,
        frame=lines,
        framesep=2mm,
        fontsize=\small
    ]{kotlin}

open class AgreementBuilder {

    // metadata
    var title: String? = null
    var introduction: String? = null

    private val clauses = mutableListOf<Clause>()

    operator fun Action.invoke(obj: Obj): ActionObject = ActionObject(this, obj)

    /**
     * Specifies that [Subject] may perform [sentence]
     */
    infix fun Subject.may(s: ActionObject): Permission {
        val permission = Permission(Allow, Sentence(this, s.action, s.object)
        clauses += permission
        return permission
    }

    /**
     * Specifies that [Subject] may NOT perform [sentence]
     */
    infix fun Subject.may(s: ActionObject): Permission {
        val permission = Permission(Forbid, Sentence(this, s.action, s.object)
        clauses += permission
        return permission
    }
    // allows declaring parties, actions, and things to use in Sentences
    val party = oneTimeProperty<Any?, Subject> { prop -> Party(prop.name) }
    val action = oneTimeProperty<Any?, Action> { prop -> Action(prop.name) }
    val thing = oneTimeProperty<Any?, Obj> { prop -> Obj.Named(prop.name) }

}
    \end{minted}
    \caption{A minimal example of the \texttt{AgreementBuilder} DSL}
    \label{fig:agreementBuilderSmall}
\end{code}


\section{Confis Agreements Samples}\label{sec:confis-agreements-samples}


\begin{code}
    \inputminted[
        autogobble,
        frame=lines,
        framesep=2mm,
        fontsize=\small
    ]
    {kotlin}{../script/src/test/resources/scripts/minimal.confis.kts}
    \caption[A minimal example of a contract]{A minimal example of a contract, writable with the \texttt{AgreementBuilder} from~\autoref{fig:agreementBuilderSmall}}
    \label{fig:confis:minimal}
\end{code}



\begin{code}
    \inputminted[
        autogobble,
        frame=lines,
        framesep=2mm,
        fontsize=\small
    ]
    {kotlin}{../script/src/test/resources/scripts/geophys.confis.kts}
    \caption{A prototype Confis representation of a seismic data license (given in~\cite{seismicDataLicence})}
    \label{fig:confis:geophys}
\end{code}

\subsection*{From Agreement, to IR, to Rules}

The following is an example of a very simple Confis Agreement:

\begin{figure}[h]
    \begin{minted}[
        autogobble,
        frame=lines,
        framesep=2mm,
    ]{kotlin}
    val alice by party
    val use by action
    val data by thing
    alice may use(data) asLongAs {
        within { (1 of June)..(30 of June) year 2022 }
    }
    \end{minted}
    \caption{Minimal Confis agreement with a circumstance}
    \label{fig:confis:min-circumstance}
\end{figure}

This clause is simple and generates a single rule when querying for a circumstance question~(\emph{`Under what circumstances may Alice use data?'}).
The clause is translated into the following IR:

\begin{code}
    \begin{minted}[
        autogobble,
        frame=lines,
        framesep=2mm,
        fontsize=\small,
    ]{haskell}
        Agreement(
            clauses = [
                    PermissionWithCircumstances(
                        allowance = Allow,
                        sentence = Sentence(
                            Party("alice"),
                            Action("use"),
                            Obj.Named("data")
                        ),
                        circumstanceAllowance = Allow,
                        circumstances = CircumstanceSet(
                            TimeRange.Key -> TimeRange(01/07/2022..30/01/2022),
                        ),
                    ),
                ],
            parties = [Party("alice")],
            title = null,
            description = null,
        )
\end{minted}
    \caption{IR of minimal Confis agreement with a circumstance from~\autoref{fig:confis:min-circumstance}}
    \label{fig:confis:min-circumstance-ir}
\end{code}

