\chapter{Confis Code Samples}\label{ch:confis-examples}

\newenvironment{code}{\captionsetup{type=figure}}{}


\section{Confis Software Archive Code Extracts}\label{sec:confis-software-code-extracts}

The following is a stripped-down version of the \texttt{AgreementBuilder} class (the full version can be found in the software archive).
It is meant ot demonstrate how normal Kotlin functions can come together inside a single class that can be bound to a script in order to write agreements like that of~\autoref{fig:confis:minimal}.
\begin{code}
    \begin{minted}[
        autogobble,
        frame=lines,
        framesep=2mm,
        fontsize=\small
    ]{kotlin}

open class AgreementBuilder {

    // metadata
    var title: String? = null
    var introduction: String? = null

    private val clauses = mutableListOf<Clause>()

    operator fun Action.invoke(obj: Obj): ActionObject = ActionObject(this, obj)

    /**
     * Specifies that [Subject] may perform [sentence]
     */
    infix fun Subject.may(s: ActionObject): Permission {
        val permission = Permission(Allow, Sentence(this, s.action, s.object)
        clauses += permission
        return permission
    }

    /**
     * Specifies that [Subject] may NOT perform [sentence]
     */
    infix fun Subject.may(s: ActionObject): Permission {
        val permission = Permission(Forbid, Sentence(this, s.action, s.object)
        clauses += permission
        return permission
    }
    // allows declaring parties, actions, and things to use in Sentences
    val party = oneTimeProperty<Any?, Subject> { prop -> Party(prop.name) }
    val action = oneTimeProperty<Any?, Action> { prop -> Action(prop.name) }
    val thing = oneTimeProperty<Any?, Obj> { prop -> Obj.Named(prop.name) }

}
    \end{minted}
    \caption{A minimal example of the \texttt{AgreementBuilder} DSL}
    \label{fig:agreementBuilderSmall}
\end{code}


\section{Confis Agreements Samples}\label{sec:confis-agreements-samples}


\begin{code}
    \inputminted[
        autogobble,
        frame=lines,
        framesep=2mm,
        fontsize=\small
    ]
    {kotlin}{../script/src/test/resources/scripts/minimal.confis.kts}
    \caption[A minimal example of a contract]{A minimal example of a contract, writable with the \texttt{AgreementBuilder} from~\autoref{fig:agreementBuilderSmall}}
    \label{fig:confis:minimal}
\end{code}



\begin{code}
    \inputminted[
        autogobble,
        frame=lines,
        framesep=2mm,
        fontsize=\small
    ]
    {kotlin}{../script/src/test/resources/scripts/geophys.confis.kts}
    \caption{A prototype Confis representation of a seismic data license (given in~\cite{seismicDataLicence})}
    \label{fig:confis:geophys}
\end{code}

\subsection*{From Agreement, to IR, to Rules}

The following is an example of a very simple Confis Agreement:

\begin{figure}[h]
    \begin{minted}[
        autogobble,
        frame=lines,
        framesep=2mm,
    ]{kotlin}
    val alice by party
    val use by action
    val data by thing
    alice may use(data) asLongAs {
        within { (1 of June)..(30 of June) year 2022 }
    }
    \end{minted}
    \caption{Minimal Confis agreement with a circumstance}
    \label{fig:confis:min-circumstance}
\end{figure}

This clause is simple and generates a single rule when querying for a circumstance question~(\emph{`Under what circumstances may Alice use data?'}).
The clause is translated into the following IR:

\begin{code}
    \begin{minted}[
        autogobble,
        frame=lines,
        framesep=2mm,
        fontsize=\small,
    ]{haskell}
        Agreement(
            clauses = [
                    PermissionWithCircumstances(
                        allowance = Allow,
                        sentence = Sentence(
                            Party("alice"),
                            Action("use"),
                            Obj.Named("data")
                        ),
                        circumstanceAllowance = Allow,
                        circumstances = CircumstanceSet(
                            TimeRange.Key -> TimeRange(01/07/2022..30/01/2022),
                        ),
                    ),
                ],
            parties = [Party("alice")],
            title = null,
            description = null,
        )
    \end{minted}
    \caption{IR of minimal Confis agreement with a circumstance from~\autoref{fig:confis:min-circumstance}}
    \label{fig:confis:min-circumstance-ir}
\end{code}

\subsubsection{Meat Contract example}

\begin{table}
    \centering
    \setlength{\fboxsep}{10pt}
    \fbox{
        \begin{minipage}{0.8\textwidth}
            \renewcommand{\labelenumii}{\theenumii}
            \renewcommand{\theenumii}{\theenumi.\arabic{enumii}.}

            This agreement is entered into as of the date <effDate>, between
            <party1> as Seller with the address <retAdd>, and <party2> as Buyer
            with the address <delAdd>.\\
            \begin{enumerate}
                \item \textbf{Payment \& Delivery}
                \begin{enumerate}
                    \item Seller shall sell an amount of <qnt> meat with <qlt> quality (“goods”) to the Buyer.
                    \item Title in the Goods shall not pass on to the Buyer until payment of the amount owed has been made in full.
                    \item The Seller shall deliver the Order in one delivery within <delDueDateDays> days to the Buyer at its warehouse.
                    \item The Buyer shall pay <amt> (“amount”) in <curr> (“currency”) to the Seller before <payDueDate>.
                    \item In the event of late payment of the amount owed due, the Buyer shall pay interests equal to <intRate> the Seller may suspend performance of all of its obligations under the agreement until payment of amounts due has been received in full.
                \end{enumerate}
                \item \textbf{Assignment}
                \begin{enumerate}
                    \item The rights and obligations are not assignable by Buyer.
                \end{enumerate}
                \item \textbf{Termination}
                \begin{enumerate}
                    \item Any delay in delivery of the goods will not entitle the Buyer to terminate the Contract unless such delay exceeds 10 Days.
                \end{enumerate}
                \item \textbf{Confidentiality}
                \begin{enumerate}
                    \item Both Seller and Buyer must keep the contents of this contract confidential during the execution of the contract and six months after the termination of the contract.
                \end{enumerate}
            \end{enumerate}
        \end{minipage}
%    \vspace{0.2cm}
    }
    \caption{Sample of a meat purchase and sales contract}
    \label{tab:meat}
\end{table}

\begin{code}
    \inputminted[
        autogobble,
        frame=lines,
        framesep=2mm,
        fontsize=\small
    ]
    {prolog}{figures/meatSales.symboleo}
    \caption{Symboleo Specification for~\nameref{tab:meat}}
    \label{fig:symbolio:meatSales}
\end{code}

