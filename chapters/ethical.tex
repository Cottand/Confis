% Chapter Template

\chapter{Ethical Considerations}\label{ch:ethical}

\section{Legal Implications}\label{sec:legal-implications}

 This project hopes to automate enforcement of clauses in legal agreements.
Contracts commonly have exceptions where violating the terms of the agreement is not a breach of the contract in
circumstances outside the control of all parties (referred to as \textit{`Force Majeure'}~\cite{forceMajeureDefinition} -
see~\cite[\textsection13.14]{jetbrainsEduLicence} for an example).\\

% TODO come back to this after writing the intro
It is still to be seen to what extent this project enforces clauses and to what extent it only provides
a decentralised audit trail.
The former may impede actions that could be law-compliant in case of force majeure, while the latter
would give parties more freedom to breach the contract (and leave it up to the other party to follow up with legal
action).

While the latter may seem like a safer option, enforcing compliance outside of courts can be very attractive to all
parties because they would be able to protect the terms of the contract without having to go to court and expending the
financial and legal resources to take action in case of contract breach.
So much so that this is the case already in many software licenses~(where~\cite{jetbrainsEduLicence} is a prime example),
where the end user has their access revoked before the licensor.
% TODO PICKUP talk about existing auto enforcing and add it to background too!!
