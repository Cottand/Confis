\chapter{Introduction}\label{ch:introduction}

\section*{The Hassle of Enterprise Legal Agreements}\label{sec:intro:legal-hassle}

Businesses are routinely bound by legal agreements to each other and those agreements give them
constraints with respect to the legal boundaries they can operate within.
While compliance with the agreement is typically in the interest of all parties involved, it is
difficult to achieve: legal advice - most commonly costly and billed by the hour - must be sought
out to deal with any non-routine situation.

Additionally, even when an agreement is being complied with by all, parties must be able to prove
they are compliant, leading to large amounts of unstructured documents of audit trails.\\

I suggest automating some hassles of legal agreement compliance, audit and verification, by encoding
a contract in a machine-readable format and enable querying it and logging interactions, while
preserving privacy.
The goals are:
\begin{itemize}
    \item Minimising costs in legal advice by allowing a business to 'ask questions' about the legal
    boundaries within which they are able to operate.
    \item Enabling secure audit trails that all parties involved in the agreement can trust in order
    to prove compliance (or lack thereof!) without the need for trusted third parties.
    \item Meeting business' privacy requirements with respect to information about the contracts
    they sign.
\end{itemize}

In order to limit the scope of work, this project will initially focus on data and software
licences, particular kinds of legal agreements where access is granted to documents existing in
digital formats (eg, bytes) and parties are likely to employ technical staff.
